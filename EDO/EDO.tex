\documentclass[12pt,a4paper,twoside]{article}
\addtolength{\textheight}{80pt} \addtolength{\topmargin}{-50pt}
\textwidth 164mm \oddsidemargin -2.25mm \evensidemargin -2.25mm
\usepackage{amssymb}
\usepackage{amsmath}
\usepackage{amsthm} % theoremes
\usepackage[T1]{fontenc}
\usepackage[utf8]{inputenc} 
\usepackage[francais]{babel}
\usepackage{enumitem}
\usepackage{url}
\usepackage{graphicx}
\usepackage{comment}
\usepackage{xcolor}

%% pour les figures
\usepackage{tikz}
\usepackage{forloop}


%% pour les exercices
%\usepackage{exercise}

\begin{document}

%%%%%%%%%%%%%%%%%%%%%%%%%%%%%%%%%%%%%
\newtheorem{theorem}{Th\'eor\`eme}
\newtheorem{proposition}{Proposition}
\newtheorem{definition}{D\'efinition}
\newcommand*{\R}{\mathbb{R}}
\newcommand*{\N}{\mathbb{N}}

\begin{center}
{\bf \Huge Analyse du sch\'ema d'Euler explicite}
\end{center}

%\vfill
%===================================================================================

%%%%%%%%%%%%%%%%%%%%%%%%%%%%%%%%%%%%%%%%%%%%%%%%%%%%%%%%%%%%%%%%%%%%%%%%%%%%%%%%%%%%
\section{Pr\'esentation du sch\'ema d'Euler explicite}

On cherche \`a approcher num\'eriquement sur l'intervalle $[t_0,T]$
une \'equation aux d\'eriv\'ees ordinaires (EDO)
de la forme
\begin{equation}
  \label{eq:pb}
  \left\{
    \begin{array}{l}
      y'(t) = f(t,y(t)) \text{ sur } [t_0,T] ,
      \\
      y(t_0) = y_0 ,
    \end{array}
  \right.
\end{equation}
o\`u $f : [t_0,T] \times \R \to \R$ et $y_0 \in \R$ sont des donn\'ees du probl\`eme.
Ici on consid\`ere que le probl\`eme est scalaire ($y(t) \in \R$),
on peut raisonner de mani\`ere analogue s'il est vectoriel ($y(t) \in \R^d$).


Pour approcher num\'eriquement le probl\`eme \eqref{eq:pb}, on d\'ecoupe l'intervalle
$[t_0,T]$ en $N$ sous-intervalles de longueur uniforme.
La longueur de ces sous-intervalles est appel\'ee pas de discr\'etisation.
On la note
\begin{align*}
  h = \frac{T - t_0}{N} .
\end{align*}
On d\'efinit ainsi une discr\'etisation 
de l'intervalle $[t_0,T]$ avec les temps $t_k = t_0 + kh$ avec $0 \leq k \leq N$.


Le sch\'ema d'Euler consiste \`a approcher la d\'eriv\'ee par un taux d'accroissement:
\begin{align}
  \label{eq:taux_acc}
  y'(t_k) \simeq \dfrac{y(t_{k+1}) - y(t_k)}{t_{k+1} - t_k } = \dfrac{y(t_{k+1}) - y(t_k)}{h} .
\end{align}
On cherche donc \`a calculer les termes d'une suite $(y_k)_{0 \leq k \leq N}$
en rempla\c{c}ant dans \eqref{eq:pb} la d\'eriv\'ee par le taux d'accroissement
\eqref{eq:taux_acc}.
Ceci revient donc \`a calculer la suite $(y_k)_{0 \leq k \leq N}$ par r\'ecurrence comme:
\begin{align*}
  \left\{
  \begin{array}{l}
    y_{k+1} = y_k + h f(t_k , y_k), \qquad 0 \leq k \leq N-1 ,
    \\
    y_0 = y(t_0) .
  \end{array}
  \right.
\end{align*}

Nous voyons que la valeur de $y_{k+1}$ peut \^etre calcul\'ee de mani\`ere explicite 
\`a partir de la valeur de $y_k$. C'est pour cela que l'on parle de sch\'ema
d'Euler explicite.


Dans l'ensemble de ce document on utilise les notations standards qui consistent
\`a noter $(y_k)$ la suite calcul\'ee par le sch\'ema d'Euler explicite
et $y(t)$ la solution exacte (solution de \eqref{eq:pb}) au temps $t \in [t_0,T]$.
Le but d'une telle approche est de construire une suite $(y_k)$ qui s'approche
des $(y(t_k))$. Nous allons maintenant \'etudier sous quelles conditions
$y_k$ est une bonne approximation de $y(t_k)$.


%%%%%%%%%%%%%%%%%%%%%%%%%%%%%%%%%%%%%%%%%%%%%%%%%%%%%%%%%%%%%%%%%%%%%%%%%%%%%%%%%%%%
\section{Analyse des sch\'emas explicites \`a un pas}

Pour introduire le sch\'ema d'Euler explicite, nous avons approch\'e une d\'eriv\'ee
par un taux d'accroissement. Cette approximation est correcte dans la limite
$h \to 0$ (ou de mani\`ere \'equivalente $N \to +\infty$). 
Nous allons maintenant d\'efinir un certain nombre de notions qui traitent 
du comportement de la suite $(y_k)$ quand $h \to 0$.
Nous exposons ces notions dans le cadre plus g\'en\'eral 
des sch\'emas explicites \`a un pas de temps.
On consid\`ere donc que la suite $(y_k)$ est g\'en\'er\'ee
par $y_0 = y(t_0)$ et
\begin{equation}
  \label{eq:1pas}
  y_{k+1} = y_k + h F(t_k,y_k,h) , \qquad (k \geq 0),
\end{equation}
o\`u $h$ est le pas de discr\'etisation et $F$ est une fonction \`a trois variables
donn\'ee qui d\'epend du sch\'ema utilis\'e.
Par exemple, dans le cas du sch\'ema d'Euler explicite on a $F(t,y,h) = f(t,y)$.


\begin{definition}[Erreur de consistance]
  On appelle erreur de consistance la quantit\'e
  \begin{align*}
    \varepsilon_k = y(t_{k+1}) - y(t_k) - h F(t_k , y(t_k) , h ) .
  \end{align*}
\end{definition}
Il s'agit de la diff\'erence entre la solution exacte au temps $t_{k+1}$
et la solution g\'en\'er\'ee par le sch\'ema \eqref{eq:1pas}
\`a partir de la solution exacte au temps $t_k$.
Cette erreur correspond donc \`a l'erreur g\'en\'er\'ee
par le sch\'ema au cours de la r\'esolution dans l'intervalle
$[t_k , t_{k+1}]$.

\begin{definition}[Consistance]
  On dit que le sch\'ema \eqref{eq:1pas} est consistant si 
  \begin{align*}
    \lim_{h \to 0} \sum_{k=0}^{N-1} | \varepsilon_k | = 0 ,
  \end{align*}
  o\`u $\varepsilon_k$ est l'erreur de consistance d\'efinie pr\'ec\'edemment.
\end{definition}
Notons que la notion de consistance se fait par rapport \`a un probl\`eme continu
(ici \eqref{eq:pb}) puisqu'on utilise la solution exacte $y(t)$.
Cette notion consiste donc \`a dire que l'erreur du sch\'ema g\'en\'er\'ee
au cours de toutes les it\'erations tend vers 0 quand le pas de discr\'etisation
tend vers 0.



\begin{definition}[Ordre]
  On dit que le sch\'ema \eqref{eq:1pas} est d'ordre $p \in \N^*$ s'il existe
  une constante $C>0$ ind\'ependante de $h$ telle que
  \begin{align*}
    \sum_{k=0}^{N-1} | \varepsilon_k | \leq C h^{p} .
  \end{align*}
\end{definition}


Notons tout d'abord que si le sch\'ema est d'ordre $p \in \N^*$, alors il est consistant.
Cette notion d'ordre permet de quantifier la consistance d'un sch\'ema:
plus l'ordre du sch\'ema est \'elev\'e plus la somme des erreurs de consistance
tend rapidement vers 0.


\begin{definition}[Stabilit\'e]
  Le sch\'ema \eqref{eq:1pas} est stable s'il existe une constante $C > 0$
  ind\'ependante de $N$ (et donc de $h$) telle que pour toute suite 
  $(\eta_k)_{0 \leq k \leq N}$, les suites $(y_k)$ et $(z_k)$ d\'efinies comme
  \begin{align*}
    y_0 \in \R \quad \text{et} \quad y_{k+1} = y_k + h F(t_k,y_k,h) ,
    \\
    z_0 = y_0 + \eta_0 \quad \text{et} \quad z_{k+1} = z_k + h F(t_k,z_k,h) + \eta_{k+1} ,
  \end{align*}
  v\'erifient
  \begin{align*}
    \sup_{0 \leq k \leq N} | y_k - z_k | \leq C \sum_{n=0}^N | \eta_n | .
  \end{align*}
\end{definition}

Cette notion correspond \`a dire que si une erreur de calcul s'introduit
(ici l'erreur de calcul est le $\eta_k$) alors la solution sera perturb\'ee
de mani\`ere "continue" par rapport \`a cette erreur de calcul.
Ici, $(y_k)$ est la solution num\'erique attendue et $(z_k)$ est la solution
num\'erique obtenue apr\`es perturbation.


En pratique, des erreurs de calcul apparaissent lors des op\'erations informatiques.
Ces erreurs sont tr\`es petites par rapport aux valeurs de la solution du probl\`eme.
Ainsi, si le sch\'ema est stable, ces erreurs sont n\'egligeables et ne d\'egradent
pas le r\'esultat final.


\begin{definition}[Convergence]
  Le sch\'ema \eqref{eq:1pas} converge (ou est convergent) si
  \begin{align*}
    \lim_{h \to 0} \sup_{0 \leq k \leq N} | y_k - y(t_k) | = 0 .
  \end{align*}
\end{definition}

La convergence d'un sch\'ema assure que les valeurs donn\'ees par celui-ci
tendent vers les valeurs exactes de l'EDO quand $h \to 0$.
Il s'agit donc de la propri\'et\'e principale que l'on recherche puisqu'elle
garantit un bon comportement du sch\'ema quand $h \to 0$.
En pratique, on d\'emontre la convergence \`a partir de la consistance
et de la stabilit\'e gr\^ace au th\'eor\`eme suivant.

\begin{theorem}
  \label{thm:Lax}
  Si le sch\'ema \eqref{eq:1pas} est consistant et stable, alors il est convergent.
  Si de plus ce sch\'ema est d'ordre $p \in \N^*$, alors il existe une constante $C>0$
  ind\'ependante de $h$ telle que
  $\sup_{0 \leq k \leq N} | y_k - y(t_k) | \leq C h^p$.
\end{theorem}

Nous voyons ici, que comme pour la notion d'ordre, on peut quantifier la convergence du sch\'ema.
En fait, plus l'ordre du sch\'ema est \'elev\'e, plus celui-ci converge vite (sous r\'eserve qu'il soit stable).

%%%%%%%%%%%%%%%%%%%%%%%%%%%%%%%%%%%%%%%%%%%%%%%%%%%%%%%%%%%%%%%%%%%%%%%%%%%%%%%%%%%%
\section{Application au sch\'ema d'Euler explicite}

On s'int\'eresse dans cette section \`a l'analyse du sch\'ema d'Euler explicite.
Nous allons montrer que, sous certaines conditions, 
ce sch\'ema est consistant et d'ordre 1 et qu'il est stable.
En appliquant le th\'eor\`eme \ref{thm:Lax}, nous savons alors qu'il v\'erifie
\begin{align*}
  \sup_{0 \leq k \leq N} | y_k - y(t_k) | \leq C h .
\end{align*}


\begin{proposition}
  On suppose que $f$ est continue sur $[t_0 , T] \times \R$,
  alors le sch\'ema d'Euler explicite est d'ordre 1.
\end{proposition}

\begin{proof}
  \'Etant donn\'ee que $f$ est continue, toute solution de \eqref{eq:pb}
  est de classe $C^1$.
  Nous pouvons alors \'ecrire le d\'eveloppement de Taylor
  \begin{align*}
    y(t_{k+1}) 
    &= y(t_k) + h y'(t_k) + O(h^2)
    \\
    &= y(t_k) + h f(t_k , y(t_k) ) + O(h^2) .
  \end{align*}
  Ainsi, l'erreur de consistance vaut
  $\varepsilon_k = y(t_{k+1}) - y(t_k) - h f(t_k,y(t_k)) = O(h^2)$.
  Donc 
  \begin{align*}
    \sum_{k=0}^{N-1} | \varepsilon_k | = O(h) ,
  \end{align*}
  et le sch\'ema est d'ordre 1.
\end{proof}


\begin{proposition}
  On suppose que $f$ est globalement Lipschitzienne par rapport \`a sa variable
  $y$ uniform\'ement en sa variable $t$, i.e. 
  $\exists L > 0, \forall y, z \in \R, \forall t \in [t_0,T], 
  | f(t,y) - f(t,z) | \leq L | y - z |$.
  Alors le sch\'ema d'Euler explicite est stable.
\end{proposition}

\begin{proof}
  On d\'efinit $(y_n)$ la solution du sch\'ema d'Euler explicite
  et $(z_n)$ une solution avec des perturbations \`a chaque pas de temps.
  \begin{align*}
    y_{k+1} = y_k + h f(t_k,y_k) , 
    \\
    z_{k+1} = z_k + h f(t_k,z_k) + \eta_{k+1} ,
  \end{align*}
  avec $z_0 = y_0 + \eta_0$.
  Ainsi, pour $0 \leq k \leq N-1$,
  \begin{align*}
    | z_{k+1} - y_{k+1} | \leq (1+Lh) |z_k - y_k| + | \eta_{k+1}| .
  \end{align*}
  En appliquant le Lemme de Gr\"onwall discret (voir par exemple wikipedia),
  on obtient
  \begin{align*}
    |z_k - y_k|
    & \leq e^{(t_k - t_0) L} |\eta_0| + \sum_{i=0}^{k-1} e^{L(t_{k}-t_{i+1})} |\eta_{i+1}|
    \\
    & \leq e^{(T-t_0) L} \sum_{i=0}^{N} |\eta_i| .
  \end{align*}
  Le sch\'ema est donc stable.
\end{proof}

Ainsi, si on se place dans le cadre du th\'eor\`eme de Cauchy--Lipschitz global
($f$ continue et globalement Lipschitzienne par rapport \`a $y$ uniform\'ement en $t$)
il existe une unique solution au probl\`eme \eqref{eq:pb} et le sch\'ema
d'Euler explicite converge vers cette solution (\`a l'ordre 1).

%===================================================================================

\end{document}