\documentclass[12pt,a4paper,twoside]{article}
\addtolength{\textheight}{80pt} \addtolength{\topmargin}{-50pt}
\textwidth 164mm \oddsidemargin -2.25mm \evensidemargin -2.25mm
\usepackage{amssymb}
\usepackage{amsmath}
\usepackage[T1]{fontenc}
\usepackage[utf8]{inputenc} 
\usepackage[francais]{babel}
\usepackage{enumitem}
\usepackage{url}
\usepackage{graphicx}
\usepackage{comment}
\usepackage{xcolor}

%% pour les figures
\usepackage{tikz}
\usepackage{forloop}

\begin{document}

%%%%%%%%%%%%%%%%%%%%%%%%%%%%%%%%%%%%%

\newcommand{\Ni}[1]{{\left\|#1\right\|}_\infty}
\newcommand*{\R}{\mathbb{R}}
% \noindent
% {\rule{\textwidth}{.2mm}}\\
% \renewcommand{\labelenumi}{(\alph{enumi})}
% \input{entete}
% {\rule{\textwidth}{.2mm}}\\

\begin{center}
{\bf \Huge Introduction aux \'equations aux d\'eriv\'ees partielles}
\end{center}

% \title{Introduction aux \'equations aux d\'eriv\'ees partielles}

%\vfill
%===================================================================================

%%%%%%%%%%%%%%%%%%%%%%%%%%%%%%%%%%%%%%%%%%%%%%%%%%%%%%%%%%%%%%%%%%%%%%%%%%%%%%%%%%%%
\section*{Avant-propos}

Le but de ce cours est de vous proposer une introduction \`a la th\'eorie
des \'equations aux d\'eriv\'ees partielles (EDP dans la suite).
Nous \'etudierons plusieurs \'equations ainsi que leur discr\'etisation
par la m\'ethode des diff\'erences finies.

Dans le cadre du programme officiel de la pr\'eparation \`a l'agr\'egation
de math\'ematiques (\'epreuve de mod\'elisation), nous aborderons notamment:
\begin{itemize}
\item des notions \'el\'ementaires portant sur les EDP classiques en dimension 1,
\item l'\'equation de transport lin\'eaire avec la m\'ethode des caract\'eristiques,
\item l'\'equation des ondes et de la chaleur; une r\'esolution par s\'erie de Fourier
  et transform\'ee de Fourier sera propos\'ee ainsi qu'une m\'ethode de s\'eparation
  des variables.
  Les aspects qualitatifs seront abord\'es.
\item les \'equations elliptiques avec l'utilisation du th\'eor\`eme de Lax--Milgram
\item des exemples de discr\'etisation des EDP en dimension 1 avec la m\'ethode des
  diff\'erences finies. L'\'etude des propri\'et\'es de ces discr\'etisations
  sera propos\'ee : notions de consistance, stabilit\'e, convergence et d'ordre.
\end{itemize}


Vous \^etes par ailleurs invit\'es \`a lire le rapport du jury (disponible sur internet).
Vous vous rendrez compte que le jury insiste notamment sur le fait que :
\begin{itemize}
\item l'\'epreuve de mod\'elisation, comme les autres, requiert de la rigueur math\'ematique,
\item il faut \'equilibrer sa pr\'esentation entre une pr\'esentation du mod\`ele \'etudi\'e,
  des preuves math\'ematiques rigoureuses, des illustrations informatiques,
\item il attend une prise de recul de la part des candidats.
  Il faudra donc notamment \^etre capable de critiquer les limites du mod\`ele pr\'esent\'e
  dans le texte, d'expliquer le comportement qualitatif de celui-ci 
  (par exemple expliquer ce qu'il se passe quand la valeur d'un param\`etre change)
  et \^etre capable de conclure sur la probl\'ematique de d\'epart.
\end{itemize}


Ce cours sera compos\'e de
\begin{itemize}
\item cinq s\'eances de cours de deux heures chacune,
\item une s\'eance de programmation de deux heures.
\end{itemize}
Dans une premi\`ere partie, nous pr\'esenterons les \'equations \'etudi\'ees
dans ce cours en donnant une id\'ee des probl\`emes physiques associ\'es.
Chacune des parties suivantes sera consacr\'ee \`a l'\'etude plus approfondie
d'une EDP. Nous y traiterons notamment les principales caract\'eristiques de cette EDP,
les outils utilis\'es pour mener des preuves ainsi qu'une discr\'etisation par diff\'erences finies.
Les EDP \'etudi\'ees dans la suite seront les \'equations elliptiques,
l'\'equation de transport, l'\'equation de la chaleur et enfin
l'\'equation des ondes.

%%%%%%%%%%%%%%%%%%%%%%%%%%%%%%%%%%%%%%%%%%%%%%%%%%%%%%%%%%%%%%%%%%%%%%%%%%%%%%%%%%%%
\section{Pr\'esentation des EDP du cours}

Nous pr\'esentons dans cette section les EDP \'etudi\'ees dans la suite du cours.
Nous essayons de donner une signification physique aux diff\'erents termes.

Nous nous int\'eressons \`a des EDP de la forme
\begin{align*}
  a \dfrac{\partial^2 u}{\partial x^2} + b \dfrac{\partial^2 u}{\partial x \partial y}
  + c \dfrac{\partial^2 u}{\partial y^2} + d \dfrac{\partial u}{\partial x}
  + e \dfrac{\partial u}{\partial y} + f u = F .
\end{align*}
Pour d\'eterminer la nature de l'EDP, on associe \`a chaque d\'eriv\'ee
la variable qui correspond \`a la direction de d\'erivation.
L'\'equation pr\'ec\'edente devient donc
\begin{align*}
  a x^2 + b xy + c y^2 + d x + e y + f = F .
\end{align*}
S'il s'agit de l'\'equation:
\begin{itemize}
\item d'une ellipse, on dira que l'\'equation est elliptique;
\item d'une parabole, on dira que l'\'equation est parabolique;
\item d'une hyperbole, on dira que l'\'equation est hyperbolique.
\end{itemize}
Cette d\'enomination n'est pas juste esth\'etique.
En effet, comme nous le verrons plus loin dans ce cours,
chacun de ces types d'\'equations dispose de propri\'et\'es sp\'ecifiques.

Notons \'egalement que les \'equations pr\'ec\'edentes d\'ependent de deux
variables d'espace. La d\'enomination pr\'ec\'edente se g\'en\'eralise 
dans le cas o\`u on aurait une seule variable ou strictement plus de deux variables.
Dans le cadre de ce cours, nous nous concentrerons sur l'\'etude d\'equations
avec une seule dimension d'espace.
On consid\'erera donc une seule variable $x$ dans le cas d'un probl\`eme stationnaire
et deux variables $t$ (le temps) et $x$ (l'espace) dans le cas d'un probl\`eme
instationaire.

Dans la suite de ce cours, on notera $\Omega$ un ouvert born\'e de $\R^d$
avec $d=1,2,3$.

%%%%%%%%%%%%%%%%%%%%%%%%%%%%%%%%%%%%%%%
\subsection{\'Equations elliptiques}

Les \'equations elliptiques apparaissent principalement dans deux contextes que nous
allons maintenant aborder.
Le premier est le cas o\`u des particules circulent dans un domaine.
Ce probl\`eme est repr\'esent\'e sur la figure \ref{fig:flux}.
Dans la partie droite de cette figure, nous nous int\'eressons \`a un probl\`eme
o\`u des particules circulent dans un milieu unidimensionnel. La position est
rep\'er\'ee par la coordonn\'ee d'espace $x$. On note $u(x)$ la densit\'e
de particules en $x$. Certaines particules entrent ou sortent du domaine
en $x$, on les note $f(x)$ le terme source les repr\'esentant.
De plus, les particules se d\'eplacent \`a travers le domaine,
on note $q(x)$ le flux de particules en $x$
(le nombre de particules qui traversent l'axe vertical d'abscisse $x$).
Ce flux est n\'egatif si les particules vont vers la gauche ($x$ d\'ecroissants)
et positif si elles vont vers la droite ($x$ croissants).

\begin{figure}
\begin{tikzpicture}[scale = 3]
  \def\h{0.1}
\def\N{11.0}
\newcounter{itx}
\newcounter{ity}

%% pente
\forloop{itx}{0}{\value{itx} < \N}{
\forloop{ity}{1}{\value{ity} < \value{itx} }{
\draw (\arabic{itx}*\h,\arabic{ity}*\h) node{$\circ$};
}
}

%% plateau
\forloop{itx}{0}{\value{itx} < 9.0}{
\forloop{ity}{1}{\value{ity} < 11.0 }{
\draw (1.0+\h+\arabic{itx}*\h,\arabic{ity}*\h) node{$\circ$};
}
}

%% axe des x
\draw[->] (0.0,0.0) -- (2.5 , 0.0);
\draw (2.5,-0.1) node{$x$};

%% terme source 1
\forloop{itx}{0}{\value{itx} < 4.0}{
\draw[->] (1.2 + 2*\h * \arabic{itx}, 1.5) -- (1.2 + 2*\h * \arabic{itx}, 1.1);
\draw (1.2 + 2*\h * \arabic{itx}, 1.3) node{$\circ$};
\draw (1.2 + 2*\h * \arabic{itx}, 1.5) node{$\circ$};
}
\draw (2.0, 1.3) node{$f(x)$};

%% terme source 2
\forloop{itx}{0}{\value{itx} < 2.0}{
\draw[->] (0.2 + 2*\h * \arabic{itx}, 0.4) -- (0.2 + 2*\h * \arabic{itx}, 0.8);
\draw (0.2 + 2*\h * \arabic{itx}, 0.6) node{$\circ$};
\draw (0.2 + 2*\h * \arabic{itx}, 0.4) node{$\circ$};
}
\draw (0.0, 0.6) node{$f(x)$};


%% densite u
\draw[<->] (2.0, \h/2.0) -- (2.0, 1.05);
\draw (2.2, 0.6) node{$u(x)$};

%% flux q
\draw[->] (0.95, 0.9) -- (0.6, 0.9);
\draw (0.8, 1.0) node{$q(x)$};
\end{tikzpicture}
\begin{tikzpicture}[scale = 2.5]
  \def\h{0.1}
\def\N{11.0}

%% rectangle
\draw (0,0) -- (2,0) -- (2,1) -- (0,1) -- (0,0);
\draw (1.0, 0.5) node{$u(x)$};

%% source term
\forloop{itx}{0}{\value{itx} < 9.0}{
\draw[->] (0.2 + 2*\h * \arabic{itx}, 1.5) -- (0.2 + 2*\h * \arabic{itx}, 1.1);
}
\draw (2.0, 1.3) node{$f(x)$};

%% flux
\draw[->] (-0.6,0.5) -- (-0.1,0.5);
\draw (-0.4, 0.3) node{$q(x)$};
\draw[->] (2.1,0.5) -- (2.8,0.5);
\draw (2.5, 0.3) node{$q(x+\delta x)$};

%% axe des abscisses
\draw[->] (-0.4,-0.1) -- (2.3,-0.1);
\draw (0.0, -0.1) -- (0.0, -0.2);
\draw (0.0, -0.3) node{$x$};
\draw (2.0, -0.1) -- (2.0, -0.2);
\draw (2.0, -0.3) node{$x+\delta x$};
\end{tikzpicture}
\caption{Un flux de particules. Gauche: repr\'esentation du probl\`eme.
  Droite: \'equilibre des flux.}
\label{fig:flux}
\end{figure}


On s'int\'eresse au cas o\`u les flux sont \`a l'\'equilibre, il n'y
a donc pas d'accumulation de particules en aucun point de l'espace.
Le probl\`eme ne d\'epend pas du temps.


Si l'on consid\`ere un \'el\'ement du domaine de taille $\delta x$ comme
sur la droite de la figure \ref{fig:flux}, le nombre de particules doit
rester constant au cours du temps.
On obtient donc la relation de conservation
$q(x) - q(x+\delta x) + f(x) \delta x = 0$,
ce qui donne
\begin{align}
  \label{eq:eq_flux}
  \dfrac{\partial q}{\partial x} = f .
\end{align}

De plus, on consid\`ere que les particules fuient les zones de forte densit\'e:
le flux $q(x)$ est orient\'e dans le sens inverse du gradient de $u$.
On note donc 
\begin{align}
  \label{eq:def_flux}
  q(x) = - k(x) \dfrac{\partial u}{\partial x}(x) . 
\end{align}
Ici $k$ est un coefficient positif qui peut d\'ependre de l'espace.
Il traduit le rapport de proportionnalit\'e entre le gradient de la densit\'e
et le flux qui en r\'esulte. Ainsi, pour une densit\'e fix\'ee,
si $k$ est grand alors les particules circuleront facilement et le flux sera important;
\`a l'inverse, un $k$ petit traduit le fait que les particules ont du mal \`a circuler dans
le milieu. On dit que $k$ est un coeeficient de diffusion.

L'\'equation finale sur $u$ est donc
\begin{align*}
  - \dfrac{\partial}{\partial x} \left(k \dfrac{\partial u}{\partial x} \right) = f .
\end{align*}

Si l'on consid\`ere plusieurs dimensions d'espace, cette \'equation devient
\begin{align*}
  - \DIV \left(k \GRAD u \right) = f ,
\end{align*}
o\`u $\DIV$ et $\GRAD$ sont respectivement les op\'erateurs divergence et gradient.


En pratique les particules peuvent r\'eellement repr\'esenter des particules physiques,
dans ce cas $u$ sera une densit\'e de particules, $q$ un flux de particules, 
$f$ un terme repr\'esentant l'apparition ou la disparition de particules et
$k$ sera un coefficient de diffusion des particules.
On peut aussi dire que les particules repr\'esentent de l'\'energie thermique.
Dans ce cas, $u$ sera la temp\'erature, $q$ un flux thermique, $f$ une source
ou un puit de chaleur et $k$ un coefficient de diffusion thermique.
Nous citerons une derni\`ere possibilit\'e selon laquelle les particules sont des
individus (humains ou animaux).
Dans ce cas, $u$ correspond \`a une densit\'e de population,
$q$ \`a un flux de population, $f$ repr\'esente des naissances ou des morts
dans la population et $k$ est un coefficient de diffusion repr\'esentant la
facilit\'e avec laquelle la population peut se d\'eplacer. 


!! TABLEAU !!


!! CONDITIONS FRONTIERES !!

Un autre cas o\`u ces \'equations apparaissent est le cas d'un mat\'eriau 
soumis \`a des contraintes m\'ecaniques.
Par exemple, on repr\'esente sur la figure \ref{fig:barre} une barre \'elastique
en \'equilibre.


\begin{figure}
\begin{tikzpicture}[scale = 1.5]
  %% barre
\draw (0,0) -- (6,0) -- (6,1) -- (0,1) -- (0,0);

%% fixation
\draw (0, -0.5) -- (0, 1.5);
\forloop{itx}{0}{\value{itx} < 8}{
\draw (0, -0.25+0.25*\arabic{itx}) -- (-0.25, -0.5+0.25*\arabic{itx});
}


%% forces
\forloop{itx}{0}{\value{itx} < 8}{
\draw[->] (0.5+0.7*\arabic{itx},0.5) node{$\times$} -- (0.9+0.7*\arabic{itx},0.5);
}
\draw (4.2,0.25) node{$f$};

%% axe des abscisses
\draw[->] (-0.5, 2.0) -- (6.0,2.0);
\draw (6.0,1.7) node{$x$};


%% barre de reference
\draw[dashed] (0.0,2.5) -- (4,2.5) -- (4,3.5) -- (0,3.5) -- (0,2.5);
\draw[dashed] (0.0,2.5) -- (0.0,1.0);
\draw[dashed] (4,2.5) -- (6,1);


%% deplacement u
\draw[dashed] (2,2.5) -- (2,-1);
\draw (2,-1.2) node{$x$};
\draw[dashed] (2,2.5) -- (2.8,1);
\draw[dashed] (2.8,1) -- (2.8,-1);
\draw[->] (2,-0.5) -- (2.8,-0.5);
\draw (2.4,-0.7) node{$u(x)$};

\end{tikzpicture}
\begin{tikzpicture}[scale = 2]
  \def\h{0.1}
\def\N{11.0}

%% rectangle 1
\draw[dashed] (-0.8,1.0) -- (1.2,1.0) -- (1.2,1.5) -- (-0.8,1.5) -- (-0.8,1.0);

%% rectangle 2
\draw (0,0) -- (2,0) -- (2,0.5) -- (0,0.5) -- (0,0);

%% axe des abscisses 1
\draw[dashed,->] (-1.2,0.9) -- (1.8,0.9);
\draw[dashed] (-0.8, 0.9) -- (-0.8, 0.8);
\draw (-0.8, 0.7) node{$x$};
\draw[dashed] (1.2, 0.9) -- (1.2, 0.8);
\draw (1.2, 0.7) node{$x+\delta x$};

%% axe des abscisses 2
\draw[->] (-0.4,-0.1) -- (2.3,-0.1);
\draw (0.0, -0.1) -- (0.0, -0.2);
\draw (0.0, -0.3) node{$x+u(x)$};
\draw (2.0, -0.1) -- (2.0, -0.2);
\draw (2.0, -0.3) node{$x + \delta x + u(x+\delta x)$};


%% autres traits
\draw[dashed] (-0.8,1.0) -- (0,0.5);
\draw[dashed] (1.2,1.0) -- (2,0.5);
\end{tikzpicture}
\caption{Un flux de particules. Gauche: repr\'esentation du probl\`eme.
  Droite: \'equilibre des flux.}
\label{fig:barre}
\end{figure}


La barre dans son \'etat initial est repr\'esent\'ee en pointill\'es.
Sous l'effet d'une force lin\'eique $f$, cette barre s'allonge
et atteint l'\'etat d'\'equilibre repr\'esent\'e en bas de la figure.
On note $u(x)$ le d\'eplacement de mati\`ere qui a eu lieu entre
l'\'etat sous la charge $f$ et l'\'etat de r\'ef\'erence en pointill\'es.


On consid\`ere que la barre est \`a l'\'equilibre m\'ecanique.
On note $q(x)$ la force qu'exerce la section de gauche sur la section de droite
en $x$. En faisant un bilan de force comme sur la partie droite de la figure \ref{fig:flux},
les forces s'exer\c{c}ant sur une portion infinit\'esimale de barre sont
la force $q(x)$ \`a gauche, la force $-q(x+\delta x)$ \`a droite
et la force lin\'eique $f(x) \delta x$. La barre \'etant \`a l'\'equilibre la somme de ces forces
est nulle. On retrouve donc \eqref{eq:eq_flux}.


De plus, la force s'exer\c{c}ant en $x$ \`a travers la section de la barre est 
proportionnelle \`a l'\'elongation de la barre et s'oppose au mouvement impos\'e.
Ceci est intuitif, pensez \`a un \'elastique si vous l'allongez il s'exerce une force
qui tend \`a le faire revenir vers sa position initiale. De plus, plus l'\'elongation
est importante, plus l'intensit\'e de la force est grande.
On obtient donc la loi d'\'elasticit\'e \eqref{eq:def_flux}
o\`u $k$ est un coefficient de raideur:
plus $k$ est grand, plus la barre est raide, plus il faut forcer pour la d\'eformer.
En pratique, le coefficient de raideur d\'epend du mat\'eriau choisi et
de la g\'eom\'etrie de la section de la barre.

Notons que dans \eqref{eq:def_flux}, la d\'eriv\'ee en espace correspond \`a 
l'\'elongation de la barre en $x$. Pour s'en convaincre, on regardera la 
partie droite de la figure \ref{fig:barre}. Un \'el\'ement de mati\`ere de longueur
$\delta x$ dans sa position de r\'ef\'erence a pour longueur 
$x + \delta x + u(x+\delta x) - u(x) - x$ sous charge $f$.
La nouvelle longueur est donc de $\delta x + u(x+\delta x) - u(x) 
\simeq \left(1 + \dfrac{\partial u}{\partial x} \right) \delta x$
et la d\'eriv\'ee partielle en $x$ est donc bien une \'elongation lin\'eique.


!! CONDITIONS FRONTIERES !!
La partie hachur\'ee \`a gauche du dessin repr\'esente le fait que la barre est encastr\'ee
dans un mur. Ainsi son d\'eplacement est forc\'ement nul \`a gauche.


!! EDP elliptique !! A faire en exo !!
%%%%%%%%%%%%%%%%%%%%%%%%%%%%%%%%%%%%%%%
\subsection{\'Equation de transport}


%%%%%%%%%%%%%%%%%%%%%%%%%%%%%%%%%%%%%%%
\subsection{\'Equation de la chaleur}

%%%%%%%%%%%%%%%%%%%%%%%%%%%%%%%%%%%%%%%
\subsection{\'Equation des ondes}

%%%%%%%%%%%%%%%%%%%%%%%%%%%%%%%%%%%%%%%%%%%%%%%%%%%%%%%%%%%%%%%%%%%%%%%%%%%%%%%%%%%%
\section{\'Equations elliptiques}

%%%%%%%%%%%%%%%%%%%%%%%%%%%%%%%%%%%%%%%%%%%%%%%%%%%%%%%%%%%%%%%%%%%%%%%%%%%%%%%%%%%%
\section{L'\'equation de transport}

%%%%%%%%%%%%%%%%%%%%%%%%%%%%%%%%%%%%%%%%%%%%%%%%%%%%%%%%%%%%%%%%%%%%%%%%%%%%%%%%%%%%
\section{L'\'equation de la chaleur}

%%%%%%%%%%%%%%%%%%%%%%%%%%%%%%%%%%%%%%%%%%%%%%%%%%%%%%%%%%%%%%%%%%%%%%%%%%%%%%%%%%%%
\section{L'\'equation des ondes}


%===================================================================================

\end{document}