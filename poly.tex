\documentclass[12pt,a4paper,twoside]{article}
\addtolength{\textheight}{80pt} \addtolength{\topmargin}{-50pt}
\textwidth 164mm \oddsidemargin -2.25mm \evensidemargin -2.25mm
\usepackage{amssymb}
\usepackage{amsmath}
\usepackage[T1]{fontenc}
\usepackage[utf8]{inputenc} 
\usepackage[francais]{babel}
\usepackage{enumitem}
\usepackage{url}
\usepackage{graphicx}
\usepackage{comment}
\usepackage{xcolor}

%% pour les figures
\usepackage{tikz}
\usepackage{forloop}

\begin{document}

%%%%%%%%%%%%%%%%%%%%%%%%%%%%%%%%%%%%%

\newcommand{\Ni}[1]{{\left\|#1\right\|}_\infty}
\newcommand*{\R}{\mathbb{R}}
% \noindent
% {\rule{\textwidth}{.2mm}}\\
% \renewcommand{\labelenumi}{(\alph{enumi})}
% \input{entete}
% {\rule{\textwidth}{.2mm}}\\

\begin{center}
{\bf \Huge Introduction aux \'equations aux d\'eriv\'ees partielles}
\end{center}

% \title{Introduction aux \'equations aux d\'eriv\'ees partielles}

%\vfill
%===================================================================================

%%%%%%%%%%%%%%%%%%%%%%%%%%%%%%%%%%%%%%%%%%%%%%%%%%%%%%%%%%%%%%%%%%%%%%%%%%%%%%%%%%%%
\section*{Avant-propos}

Le but de ce cours est de vous proposer une introduction \`a la th\'eorie
des \'equations aux d\'eriv\'ees partielles (EDP dans la suite).
Nous \'etudierons plusieurs \'equations ainsi que leur discr\'etisation
par la m\'ethode des diff\'erences finies.

Dans le cadre du programme officiel de la pr\'eparation \`a l'agr\'egation
de math\'ematiques (\'epreuve de mod\'elisation), nous aborderons notamment:
\begin{itemize}
\item des notions \'el\'ementaires portant sur les EDP classiques en dimension 1,
\item l'\'equation de transport lin\'eaire avec la m\'ethode des caract\'eristiques,
\item l'\'equation des ondes et de la chaleur; une r\'esolution par s\'erie de Fourier
  et transform\'ee de Fourier sera propos\'ee ainsi qu'une m\'ethode de s\'eparation
  des variables.
  Les aspects qualitatifs seront abord\'es.
\item les \'equations elliptiques avec l'utilisation du th\'eor\`eme de Lax--Milgram
\item des exemples de discr\'etisation des EDP en dimension 1 avec la m\'ethode des
  diff\'erences finies. L'\'etude des propri\'et\'es de ces discr\'etisations
  sera propos\'ee : notions de consistance, stabilit\'e, convergence et d'ordre.
\end{itemize}


Vous \^etes par ailleurs invit\'es \`a lire le rapport du jury (disponible sur internet).
Vous vous rendrez compte que le jury insiste notamment sur le fait que :
\begin{itemize}
\item l'\'epreuve de mod\'elisation, comme les autres, requiert de la rigueur math\'ematique,
\item il faut \'equilibrer sa pr\'esentation entre une pr\'esentation du mod\`ele \'etudi\'e,
  des preuves math\'ematiques rigoureuses, des illustrations informatiques,
\item il attend une prise de recul de la part des candidats.
  Il faudra donc notamment \^etre capable de critiquer les limites du mod\`ele pr\'esent\'e
  dans le texte, d'expliquer le comportement qualitatif de celui-ci 
  (par exemple expliquer ce qu'il se passe quand la valeur d'un param\`etre change)
  et \^etre capable de conclure sur la probl\'ematique de d\'epart.
\end{itemize}


Ce cours sera compos\'e de
\begin{itemize}
\item cinq s\'eances de cours de deux heures chacune,
\item une s\'eance de programmation de deux heures.
\end{itemize}
Dans une premi\`ere partie, nous pr\'esenterons les \'equations \'etudi\'ees
dans ce cours en donnant une id\'ee des probl\`emes physiques associ\'es.
Chacune des parties suivantes sera consacr\'ee \`a l'\'etude plus approfondie
d'une EDP. Nous y traiterons notamment les principales caract\'eristiques de cette EDP,
les outils utilis\'es pour mener des preuves ainsi qu'une discr\'etisation par diff\'erences finies.
Les EDP \'etudi\'ees dans la suite seront les \'equations elliptiques,
l'\'equation de transport, l'\'equation de la chaleur et enfin
l'\'equation des ondes.

%%%%%%%%%%%%%%%%%%%%%%%%%%%%%%%%%%%%%%%%%%%%%%%%%%%%%%%%%%%%%%%%%%%%%%%%%%%%%%%%%%%%
\section{Pr\'esentation des EDP du cours}

Nous pr\'esentons dans cette section les EDP \'etudi\'ees dans la suite du cours.
Nous essayons de donner une signification physique aux diff\'erents termes.

Nous nous int\'eressons \`a des EDP de la forme
\begin{align*}
  a \dfrac{\partial^2 u}{\partial x^2} + b \dfrac{\partial^2 u}{\partial x \partial y}
  + c \dfrac{\partial^2 u}{\partial y^2} + d \dfrac{\partial u}{\partial x}
  + e \dfrac{\partial u}{\partial y} + f u = F .
\end{align*}
Pour d\'eterminer la nature de l'EDP, on associe \`a chaque d\'eriv\'ee
la variable qui correspond \`a la direction de d\'erivation.
L'\'equation pr\'ec\'edente devient donc
\begin{align*}
  a x^2 + b xy + c y^2 + d x + e y + f = F .
\end{align*}
S'il s'agit de l'\'equation:
\begin{itemize}
\item d'une ellipse, on dira que l'\'equation est elliptique;
\item d'une parabole, on dira que l'\'equation est parabolique;
\item d'une hyperbole, on dira que l'\'equation est hyperbolique.
\end{itemize}
Cette d\'enomination n'est pas juste esth\'etique.
En effet, comme nous le verrons plus loin dans ce cours,
chacun de ces types d'\'equations dispose de propri\'et\'es sp\'ecifiques.

Notons \'egalement que les \'equations pr\'ec\'edentes d\'ependent de deux
variables d'espace. La d\'enomination pr\'ec\'edente se g\'en\'eralise 
dans le cas o\`u on aurait une seule variable ou strictement plus de deux variables.
Dans le cadre de ce cours, nous nous concentrerons sur l'\'etude d\'equations
avec une seule dimension d'espace.
On consid\'erera donc une seule variable $x$ dans le cas d'un probl\`eme stationnaire
et deux variables $t$ (le temps) et $x$ (l'espace) dans le cas d'un probl\`eme
instationaire.

Dans la suite de ce cours, on notera $\Omega$ un ouvert born\'e de $\R^d$
avec $d=1,2,3$.

%%%%%%%%%%%%%%%%%%%%%%%%%%%%%%%%%%%%%%%
\subsection{\'Equations elliptiques}

Les \'equations elliptiques apparaissent principalement dans deux contextes que nous
allons maintenant aborder.
Le premier est le cas o\`u des particules circulent dans un domaine.

\begin{tikzpicture}[scale = 3]
  \def\h{0.1}
\def\N{11.0}
\newcounter{itx}
\newcounter{ity}

%% pente
\forloop{itx}{0}{\value{itx} < \N}{
\forloop{ity}{1}{\value{ity} < \value{itx} }{
\draw (\arabic{itx}*\h,\arabic{ity}*\h) node{$\circ$};
}
}

%% plateau
\forloop{itx}{0}{\value{itx} < 9.0}{
\forloop{ity}{1}{\value{ity} < 11.0 }{
\draw (1.0+\h+\arabic{itx}*\h,\arabic{ity}*\h) node{$\circ$};
}
}

%% axe des x
\draw[->] (0.0,0.0) -- (2.5 , 0.0);
\draw (2.5,-0.1) node{$x$};

%% terme source 1
\forloop{itx}{0}{\value{itx} < 4.0}{
\draw[->] (1.2 + 2*\h * \arabic{itx}, 1.5) -- (1.2 + 2*\h * \arabic{itx}, 1.1);
\draw (1.2 + 2*\h * \arabic{itx}, 1.3) node{$\circ$};
\draw (1.2 + 2*\h * \arabic{itx}, 1.5) node{$\circ$};
}
\draw (2.0, 1.3) node{$f(x)$};

%% terme source 2
\forloop{itx}{0}{\value{itx} < 2.0}{
\draw[->] (0.2 + 2*\h * \arabic{itx}, 0.4) -- (0.2 + 2*\h * \arabic{itx}, 0.8);
\draw (0.2 + 2*\h * \arabic{itx}, 0.6) node{$\circ$};
\draw (0.2 + 2*\h * \arabic{itx}, 0.4) node{$\circ$};
}
\draw (0.0, 0.6) node{$f(x)$};


%% densite u
\draw[<->] (2.0, \h/2.0) -- (2.0, 1.05);
\draw (2.2, 0.6) node{$u(x)$};

%% flux q
\draw[->] (0.95, 0.9) -- (0.6, 0.9);
\draw (0.8, 1.0) node{$q(x)$};
\end{tikzpicture}
\begin{tikzpicture}[scale = 2.5]
  \def\h{0.1}
\def\N{11.0}

%% rectangle
\draw (0,0) -- (2,0) -- (2,1) -- (0,1) -- (0,0);
\draw (1.0, 0.5) node{$u(x)$};

%% source term
\forloop{itx}{0}{\value{itx} < 9.0}{
\draw[->] (0.2 + 2*\h * \arabic{itx}, 1.5) -- (0.2 + 2*\h * \arabic{itx}, 1.1);
}
\draw (2.0, 1.3) node{$f(x)$};

%% flux
\draw[->] (-0.6,0.5) -- (-0.1,0.5);
\draw (-0.4, 0.3) node{$q(x)$};
\draw[->] (2.1,0.5) -- (2.8,0.5);
\draw (2.5, 0.3) node{$q(x+\delta x)$};

%% axe des abscisses
\draw[->] (-0.4,-0.1) -- (2.3,-0.1);
\draw (0.0, -0.1) -- (0.0, -0.2);
\draw (0.0, -0.3) node{$x$};
\draw (2.0, -0.1) -- (2.0, -0.2);
\draw (2.0, -0.3) node{$x+\delta x$};
\end{tikzpicture}


%%%%%%%%%%%%%%%%%%%%%%%%%%%%%%%%%%%%%%%
\subsection{\'Equation de transport}


%%%%%%%%%%%%%%%%%%%%%%%%%%%%%%%%%%%%%%%
\subsection{\'Equation de la chaleur}

%%%%%%%%%%%%%%%%%%%%%%%%%%%%%%%%%%%%%%%
\subsection{\'Equation des ondes}

%%%%%%%%%%%%%%%%%%%%%%%%%%%%%%%%%%%%%%%%%%%%%%%%%%%%%%%%%%%%%%%%%%%%%%%%%%%%%%%%%%%%
\section{\'Equations elliptiques}

%%%%%%%%%%%%%%%%%%%%%%%%%%%%%%%%%%%%%%%%%%%%%%%%%%%%%%%%%%%%%%%%%%%%%%%%%%%%%%%%%%%%
\section{L'\'equation de transport}

%%%%%%%%%%%%%%%%%%%%%%%%%%%%%%%%%%%%%%%%%%%%%%%%%%%%%%%%%%%%%%%%%%%%%%%%%%%%%%%%%%%%
\section{L'\'equation de la chaleur}

%%%%%%%%%%%%%%%%%%%%%%%%%%%%%%%%%%%%%%%%%%%%%%%%%%%%%%%%%%%%%%%%%%%%%%%%%%%%%%%%%%%%
\section{L'\'equation des ondes}


%===================================================================================

\end{document}